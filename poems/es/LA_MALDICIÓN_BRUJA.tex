\documentclass[11pt, a4paper]{article} % Document font size and paper size
\usepackage[dvipsnames,svgnames,table]{xcolor}
\usepackage{verse} % Required for typesetting poems - this package drives this template
\usepackage{graphicx}
\usepackage[margin=3.15cm,footskip=0.25cm]{geometry}
\usepackage{epstopdf}
\usepackage{hyperref}
\usepackage{amsmath}
\usepackage{amssymb}
\usepackage[utf8]{inputenc}
\usepackage[spanish]{babel}
\usepackage{xcolor}
\usepackage{pagecolor}
\usepackage[normalem]{ulem}

\usepackage{palatino} % Use the Palatino font by default
%\usepackage{stix} % Alternative Stix font
\setlength{\parindent}{0pt} % Disable paragraph indentation
% Author styles
\newcommand{\poemauthorcenter}[1]{\nopagebreak{\centering\footnotesize\textsc{#1}\par}} % Author as a footnote at the end of the poem center aligned
\newcommand{\poemauthorright}[1]{\nopagebreak{\raggedleft\footnotesize\textsc{#1}\par}} % Author as a footnote at the end of the poem aligned right
\renewcommand{\poemtitlefont}{\normalfont\bfseries\large\centering} % Define the poem title style
\setlength{\stanzaskip}{0.75\baselineskip} % The distance between stanzas
\pagestyle{empty} % Stop page numbering through the document
%\pagecolor{yellow!30!orange}
\pagecolor{Tan!30!white}
% \pagecolor{white}


%%


% new LaTeX commands
\begin{document}
\poemtitle{\large LA MALDICIÓN BRUJA}

\settowidth{\versewidth}{Quiere mi tiempo libre la sutil, flexible dicha de buscarla, ------------------------------------------------------------------} % Insert one of the average-sized verses, used for centering the poem

\begin{verse}[\versewidth]
{\large
%------------------------------------------------

% ---- ----
\scriptsize{
El Diablo está que me mata, si el diablo me mata quién va a revivirlo luego cuando
haga falta probar nuevamente la libertad de las criaturas. Y si descuido en gracia a 
la Diabla dejo de existir. \\
% ---- ----
Entonces vine a hablarle, a la bruja, a la diabla, a la discreta soberana muerte 
que si me lleva antes de verla construya el hilo de lo que olvidamos juntos. 
---¿por qué será que veo tan bien en la oscuridad? 
y nuestra historia es tal qué dies veces mil años hubo de estar en cuevas, 
y entre ustedes vive el dón que ve en la oscuridad tan bien como vér en los años 
y que a mi me pide permanecer por encontrarla o permanecer la memoria de la luz 
que no es como en los ojos sino luz como en los vivos ---
años ciegos vio la tierra donde fuimos cueva y donde hacía falta 
poder ver a ciegas el hombre corecto en la noche ---
y ahora hay tanta luz en los ojos que me he quedado solo. 
}
% ---- ----

% ---- ----

% ---- ----

%------------------------------------------------
}
\end{verse}

%------------------------------------------------

\poemauthorright{waajacu} % Right-aligned author

%----------------------------------------------------------------------------------------


\end{document}
