\documentclass[11pt, a4paper]{article} % Document font size and paper size
\usepackage[dvipsnames,svgnames,table]{xcolor}
\usepackage{verse} % Required for typesetting poems - this package drives this template
\usepackage{graphicx}
\usepackage{epstopdf}
\usepackage{hyperref}
\usepackage{amsmath}
\usepackage{amssymb}
\usepackage[utf8]{inputenc}
\usepackage[spanish]{babel}
\usepackage{xcolor}
\usepackage{pagecolor}
\usepackage[normalem]{ulem}

\usepackage{palatino} % Use the Palatino font by default
%\usepackage{stix} % Alternative Stix font
\setlength{\parindent}{0pt} % Disable paragraph indentation
% Author styles
\newcommand{\poemauthorcenter}[1]{\nopagebreak{\centering\footnotesize\textsc{#1}\par}} % Author as a footnote at the end of the poem center aligned
\newcommand{\poemauthorright}[1]{\nopagebreak{\raggedleft\footnotesize\textsc{#1}\par}} % Author as a footnote at the end of the poem aligned right
\renewcommand{\poemtitlefont}{\normalfont\bfseries\large\centering} % Define the poem title style
\setlength{\stanzaskip}{0.75\baselineskip} % The distance between stanzas
\pagestyle{empty} % Stop page numbering through the document
%\pagecolor{yellow!30!orange}
\pagecolor{Tan!30!white}


%%


% new LaTeX commands
\newfont{\letterbeta}{yinit.tfm}

\begin{document}

{\letterbeta}.


\poemtitle{LA RECTA REBELDÍA CONJUNTA}

\settowidth{\versewidth}{DULCE DICHA OCULTA DE LOS QUE SE REBELAN SÓLOS } % Insert one of the average-sized verses, used for centering the poem

\begin{verse}[\versewidth]
{\scriptsize

%------------------------------------------------

¿Qué generación escapa la deliciosa costumbre de rebelarse? \\
Juntos de cara al opresor, todo lo demás a nosotros mismos, \\
El poder viene de la ilusión de ser poderosos y de estar vivo, \\
Hermano humano de potentes ojos; vé que tu arma no dispare. \\
Por ser criaturas nostra vibra conjunta tiene enfermo al mundo, \\
Sucede os, que asilados en nuestras casas no podemos sembrar. \\
Nada tenemos, incluso el poder es una muy pesada larga ilusion. \\
Lo demás es estar vivos y ni la infame muerte puede matar os. \\
Yo que pronto diré de normas, sabed os que sólo tengo éste mi 
poder de poesía; no sé de los demás observadores, ni de la historia 
---que sólo vén mis ojos hasta el horizonte y de suerte vivo sin hambre, 
no puedo moverme mucho; sospecho en otros la culpa de que no 
han visto os ojos provecho de nostros esfuerzos, ni mis ojos las células, 
ni lejos las estrellas;
Pobres por desalojo de recursos son todas las viejas patrias que alcanza éste
\newline mi bello castellano imbécil. \\!

---¿por qué a nosotros no se nos permite participar de la justa causa conjunta? 
---¿porque los ganster de mi tierra no protegen al pueblo? 
---no tiene congreso de ciencia mi patria, y de nuevo ¿quién droga amenudo a mis governantes? 
que gobiernan sin notar siquiera que estan drogados ---les hace falta un congreso de ciencia, 
a mi una senda más recta. 

Es muy una norma inmunda asignar valor y dueño al territorio. \\
Norma inmunda no servir medios para criar corazones potentes. \\
Norma inmunda es también el artificio de la deuda individual. \\
Inmundo es el soberano que gobernando olvida delegar sentido.\\
Es lo más inmundo descuidar el propósito conjunto de especie.\\
Es especialmente inmunda la arquitectura del interior de las todas las casas y selvas; Todos vivimos sólos, 
atormentados por el asilo de vivir en familias pequeñas; encerrados sin vista al horizonte ni espacio 
propio para correr. Mientras tanto en lo demás del territorio no ocurre nada, sólo sirve a la ilusión 
desatendida de poseer lugares que ya yo no vigilan tus ojos y ahora temes recorrerla dezcalso.\\!

Tradicion inmunda es culminar por consumir a una criatura consciente. \\
Tradición inmunda es también la regular cobardía de los que temen al otro. \\!

La culpa de los científicos paga la investigación de la esperanza. \\
Así aún la ciencia no puede lo que debe sanar mi lengua patria. \\  
Mi patria no toma excusas, sólo sufre gobernantes mal drogados y tiene una 
población deshabida en el queacer de sí mismos; Pueblo de mi patria, que ya de 
patrias no sabe más; tenemos poca conciencia de grupo y somos insenscibles con 
la tragedia porque después de 40 mil años de guerra ¿ahora sólo quedamos vivos los cobardes?.\\!

Escuchad de a poco; mataron a los dioses y quemaron el conocimiento. \\
Sembremos la Tundra de nuestros páramos y la non-nostra ayahuasca.\\!

La guerra no es una tradicion inmunda, guarda os más respeto a los ancestros, \\
Suerte que mi tiempo no tiene más hambre, ni frio mis noches, ni sed mis días. \\
Nostra lucha no es nuestra, sino de nostra descendencia y de todo lo que sufre. \\

De los que soñando vieron prematura la muerte por nostra justa causa conjunta. \\
Debemos os hacer bien, suficiente ---producir todo lo que necesita la bella vida. \\
¿Porqué sucede que la infelicidad no despierta señales de coraje desmedido? \\!

%------------------------------------------------
}
\end{verse}

%------------------------------------------------

\poemauthorright{FUNDACIÓN DE FUENTE LIBRE WAAJACU} % Right-aligned author

%----------------------------------------------------------------------------------------


\end{document}
