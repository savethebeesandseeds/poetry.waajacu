\documentclass[11pt, a4paper]{article} % Document font size and paper size
\usepackage[dvipsnames,svgnames,table]{xcolor}
\usepackage{verse} % Required for typesetting poems - this package drives this template
\usepackage{graphicx}
\usepackage{epstopdf}
\usepackage{hyperref}
\usepackage{amsmath}
\usepackage{amssymb}
\usepackage[utf8]{inputenc}
\usepackage[spanish]{babel}
\usepackage{xcolor}
\usepackage{pagecolor}
\usepackage[normalem]{ulem}

\usepackage{palatino} % Use the Palatino font by default
%\usepackage{stix} % Alternative Stix font
\setlength{\parindent}{0pt} % Disable paragraph indentation
% Author styles
\newcommand{\poemauthorcenter}[1]{\nopagebreak{\centering\footnotesize\textsc{#1}\par}} % Author as a footnote at the end of the poem center aligned
\newcommand{\poemauthorright}[1]{\nopagebreak{\raggedleft\footnotesize\textsc{#1}\par}} % Author as a footnote at the end of the poem aligned right
\renewcommand{\poemtitlefont}{\normalfont\bfseries\large\centering} % Define the poem title style
\setlength{\stanzaskip}{0.75\baselineskip} % The distance between stanzas
\pagestyle{empty} % Stop page numbering through the document
%\pagecolor{yellow!30!orange}
\pagecolor{Tan!30!white}


%%


% new LaTeX commands
\newfont{\letterbeta}{yinit.tfm}

\begin{document}

{\letterbeta}.

\poemtitle{DESCANSANDO LA FIEBRE EN UNOS OJOS}

\settowidth{\versewidth}{} % Insert one of the average-sized verses, used for centering the poem

\begin{verse}[\versewidth]
{\scriptsize

%------------------------------------------------

Más a menudo soy sincero cuando escribo, \\
Acaso la riquisima miel en mi compañía ---versa mejor de lo que se comporta.\\
Te costará encontrar a alguien tan recto y que tan 
a menudo equivoque sus pasos, 
---vé mi mirada en ti ganas de mi y aún vé a mi cuerpo caminar lejos de ti. \\
Escaso, no busco más el cálido abrazo, ni practico más el beso de necio. \\
Me asecha impura la divinidad y mi genio holgazan me susurra vivo en 
paradoja por un potente yo y un escaso tú. \\!   

%------------------------------------------------

Dame esos ojos, ---te cambio la medida de su valor por un oro blanco que no brilla, \\
¿vienes?, ---ven llena, y desea la más fina causa; ven a cosechar el deseo en ciencia 
de la inmortalidad, que la riqueza del mundo es poca frente a la dicha de ambos. \\
Déjame conocer el despacio ritmo de tu cadera y veme fallarte, 
derrocharé mi humanidad en tí y mi grandilocuencia en los pobres. \\
Te mantendré cerca y a mi diestra, si tu corazón vé como 
desalineados iguales a los escribanos, reyes, magos, dioses y duendes.\\
Antes que digas no, ven ésta noche a conócerme; 
contigo intentarlo me basta.\\!

%------------------------------------------------
}
\end{verse}

%------------------------------------------------

\poemauthorright{FUNDACIÓN DE FUENTE LIBRE WAAJACU} % Right-aligned author

%----------------------------------------------------------------------------------------


\end{document}
