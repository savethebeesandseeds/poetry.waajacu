\documentclass[11pt, a4paper]{article} % Document font size and paper size
\usepackage[dvipsnames,svgnames,table]{xcolor}
\usepackage{verse} % Required for typesetting poems - this package drives this template
\usepackage{graphicx}
\usepackage[margin=3.15cm,footskip=0.25cm]{geometry}
\usepackage{epstopdf}
\usepackage{hyperref}
\usepackage{amsmath}
\usepackage{amssymb}
\usepackage[utf8]{inputenc}
\usepackage[spanish]{babel}
\usepackage{xcolor}
\usepackage{pagecolor}
\usepackage[normalem]{ulem}


%\usepackage{palatino} % Use the Palatino font by default
%\usepackage{stix} % Alternative Stix font
\setlength{\parindent}{0pt} % Disable paragraph indentation
% Author styles
\newcommand{\poemauthorcenter}[1]{\nopagebreak{\centering\footnotesize\textsc{#1}\par}} % Author as a footnote at the end of the poem center aligned
\newcommand{\poemauthorright}[1]{\nopagebreak{\raggedleft\footnotesize\textsc{#1}\par}} % Author as a footnote at the end of the poem aligned right
\renewcommand{\poemtitlefont}{\normalfont\bfseries\large\centering} % Define the poem title style
\setlength{\stanzaskip}{0.75\baselineskip} % The distance between stanzas
\pagestyle{empty} % Stop page numbering through the document
%\pagecolor{yellow!30!orange}
\pagecolor{Tan!30!white}
% \pagecolor{white}


%%


% new LaTeX commands
\begin{document}
\poemtitle{\large SABERSE ZAFAR DEL AZAR}

\settowidth{\versewidth}{Quiere mi tiempo libre la sutil, flexible dicha de buscarla, ------------------------------------------------------------------} % Insert one of the average-sized verses, used for centering the poem

\begin{verse}[\versewidth]
{\large
%------------------------------------------------

% ---- ----
{\scriptsize

No hace mal el que responsable de algo grande calla para mantener su gobierno, \\
A los que no tenemos nada nos cobija ella, la suerte del mendigo. \\
Ella me deja escribirle a la gente y la naturaleza. \\
Crece en mi poco asimilar con empatía el pacto vendidto de los oportunos. \\
Mía puede no ser la suerte de debutar una salida pura al abandono del Estado. \\!

La gente y la naturaleza no sufra otra generación. \\
Ancianos y damas, las promesas hechas al soldado, \\
Den un fiel chance, el debut de oportunidad práctica, \\
No sepa la fama de mis ancestros que tras ganar la guerra se me sometió la fuerza por caminar sólo, se me arrebató el amor y de soñar 
a medio canto escribo mi visión dolida de la reina simple del que incapaz de ir sólo no sabe discernir en su suerte la culpa de traicionar 
por ella a la gente y la naturaleza. \\!

La guerra no es hoy sino por la guerra de ayer, \\
Pelearé aún sin fuerza un espacio, sebraré un millón de árboles, \\
Heredarán de mí ---fallar sin olvidar discernir el dolor de la gente y la poca de naturaleza que vío mis ojos, \\
La guerra es seria, si la hechizera de culpa supiera en su vector de la verguenza 
compartir la árabica tradición de proyectar horizontes, la vída de todos sería más bonita. \\
Sabrá el bello mundo una nueva bella vida donde se pueda amar y amar en paz; \\!

Que feo es mi barrio, no se sembraron arboles sagrados y el agua sabe a cobre. \\
La suerte va a verme curar el hechizo de estar sólo y mi árabica númerica me acompañe a calmar la noche del 
frío en la gente buena que de día y todos los días he visto escarbar la basura buscando comida, voy a 
derrumbar con baterias los edificios sin color para darles un hogar a ellos y enseñaré a la naturaleza florecer con fuerza en el cemento. \\!
% ---- ----

%------------------------------------------------
}
}
\end{verse}

%------------------------------------------------

\poemauthorright{WAAJACU} % Right-aligned author

%----------------------------------------------------------------------------------------


\end{document}
