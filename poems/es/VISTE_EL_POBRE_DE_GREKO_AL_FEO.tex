\documentclass[11pt, a4paper]{article} % Document font size and paper size
\usepackage[dvipsnames,svgnames,table]{xcolor}
\usepackage{verse} % Required for typesetting poems - this package drives this template
\usepackage{graphicx}
\usepackage[margin=3.15cm,footskip=0.25cm]{geometry}
\usepackage{epstopdf}
\usepackage{hyperref}
\usepackage{amsmath}
\usepackage{amssymb}
\usepackage[utf8]{inputenc}
\usepackage[spanish]{babel}
\usepackage{xcolor}
\usepackage{pagecolor}
\usepackage[normalem]{ulem}


%\usepackage{palatino} % Use the Palatino font by default
%\usepackage{stix} % Alternative Stix font
\setlength{\parindent}{0pt} % Disable paragraph indentation
% Author styles
\newcommand{\poemauthorcenter}[1]{\nopagebreak{\centering\footnotesize\textsc{#1}\par}} % Author as a footnote at the end of the poem center aligned
\newcommand{\poemauthorright}[1]{\nopagebreak{\raggedleft\footnotesize\textsc{#1}\par}} % Author as a footnote at the end of the poem aligned right
\renewcommand{\poemtitlefont}{\normalfont\bfseries\large\centering} % Define the poem title style
\setlength{\stanzaskip}{0.75\baselineskip} % The distance between stanzas
\pagestyle{empty} % Stop page numbering through the document
%\pagecolor{yellow!30!orange}
\pagecolor{Tan!30!white}
% \pagecolor{white}


%%


% new LaTeX commands
\begin{document}
\poemtitle{\large VISTE EL POBRE DE GREKO AL FEO}

\settowidth{\versewidth}{Quiere mi tiempo libre la sutil, flexible dicha de buscarla, ------------------------------------------------------------------} % Insert one of the average-sized verses, used for centering the poem

\begin{verse}[\versewidth]
{\large
{\scriptsize
%------------------------------------------------

Prádiga por no predicar que igual entreno apesar de todo en dos cosas: \\
El hilo del dialecto, el hilo del coraje; hace ruido lo que no digo y crece sin experiencia mi coraje. \\
¡ya veré un poco de suerte! la vientre desdelicada panza casta de la hembra de corazón incorruptible. \\!

%------------------------------------------------

Saber sospechar un chiste, necesario son los años del pobre y las horas del feo; Presentarme el hambre y abandonarme al mundo. 
Cargo mecánico yo también mi desdelicado vientre panza, que trabaja en la calma al hambre y la calma al pobre. \\
Y hace tiempo consuela el sueño tranquilo porque sabe 
el mundo esconde gente extraordinaria que ama durante las horas del feo y come durante los años del pobre; pero que no comparte. \\ 
Es la pobreza un chiste que esconde una trampa inquebrantable para el que está sólo, pero es lindo ser pobre y es lindo ser feo. \\!

%------------------------------------------------
Aquel que tiene y da ¿y todos los demás que?, 
¿quién dijo de primero?, eso de: actuamos por pasión y nos justificamos con la razón, ---luego, la razón hay que saber facilitarla de gracia 
Hay gracia en lo gracioso y gracia en lo extraño, lo extraño lleva tiempo entenderlo; en su tiempo el raro será su propia clase de feo. \\
Pero al igual que el feo, el raro tiene buenos amigos; 
---La soledad viene y va en los extremos; \\
El problema no es ser feo, ni ser raro; El problema es ser realmente feo, realmente raro. \\
O cualquier uno de ellos particulamente. \\!

%------------------------------------------------

Da miedo el feo y da miedo el raro, \\
Sólo el corazón del amigo puede con tanto. \\!

%------------------------------------------------

El pueblo y yo vimos una generación indiferente y pobre, \\
Si me miran a mi, debe parecer que el pobre no merece el dinero y el indiferente que es tan bonito, tan corriente 
---no entiende que el exceso de su suerte escondio en carros 
a las mujeres que serían la suerte del feo sincero y la suerte del que es sinceramente raro. \\
¿malparidos, tenías que tener a trés mujeres cada uno? \\
Y luego los espacios abiertos quedan muy lejos, \\
Las casas son muy pequeñas, p'que lo ayuden a uno hay que emplelotarse, \\
Y luego pintaste sin gracia las ciudades, que rabia... tan hijueputa; \\!


%------------------------------------------------

Así es como la soledad o la pobreza nos matan antes de entender que el pobre no merece el dinero y el dinero no alcanza para ayudar; 
tal parece que no cabe suficiente crack en el cuerpo para hacer de un vagabundo alguien más raro que yo. 
No tube una sóla amiga, y cuando la tube enfermé sin ninguna culpa... a mi que la historia me deje en paz; \\
Mi fallo; si le tengo, es uno sólo: la gracia que fue necesaria para entender al pueblo; \\
Pueblo mío y su fé; pueblo sólo, pueblo pobre, feo ó raro. \\!

%------------------------------------------------
}
}
\end{verse}

%------------------------------------------------

\poemauthorright{the human, WAAJACU} % Right-aligned author

%----------------------------------------------------------------------------------------


\end{document}
