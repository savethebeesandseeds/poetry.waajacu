\documentclass[11pt, a4paper]{article} % Document font size and paper size
\usepackage[dvipsnames,svgnames,table]{xcolor}
\usepackage{verse} % Required for typesetting poems - this package drives this template
\usepackage{graphicx}
\usepackage{epstopdf}
\usepackage{hyperref}
\usepackage{amsmath}
\usepackage{amssymb}
\usepackage[utf8]{inputenc}
\usepackage[spanish]{babel}
\usepackage{xcolor}
\usepackage{pagecolor}
\usepackage[normalem]{ulem}

\usepackage{palatino} % Use the Palatino font by default
%\usepackage{stix} % Alternative Stix font
\setlength{\parindent}{0pt} % Disable paragraph indentation
% Author styles
\newcommand{\poemauthorcenter}[1]{\nopagebreak{\centering\footnotesize\textsc{#1}\par}} % Author as a footnote at the end of the poem center aligned
\newcommand{\poemauthorright}[1]{\nopagebreak{\raggedleft\footnotesize\textsc{#1}\par}} % Author as a footnote at the end of the poem aligned right
\renewcommand{\poemtitlefont}{\normalfont\bfseries\large\centering} % Define the poem title style
\setlength{\stanzaskip}{0.75\baselineskip} % The distance between stanzas
\pagestyle{empty} % Stop page numbering through the document
%\pagecolor{yellow!30!orange}
\pagecolor{Tan!30!white}


%%


% new LaTeX commands
\newfont{\letterbeta}{yinit.tfm}

\begin{document}

{\letterbeta}.

\poemtitle{DOS DÓCILES MANOS HELADAS}

\settowidth{\versewidth}{mil noches} % Insert one of the average-sized verses, used for centering the poem

\begin{verse}[\versewidth]
{\scriptsize

%------------------------------------------------

A un ritmo lento un tiempo rápido que no marchite y extrañemos menos el sonido de estar a solas. \\
Le mentí mil noches heladas, ahora cuando no tube nada a dos manos sinceras. \\
¿Quién siente suficiente para entregar la vida? \\
Mantenerme libre descuida tu dicha, tu gracia, tu helada escasa, cálida forma fugaz de cariño dócil incondicional. \\
Y la luna roja el instrumento de mi partida, pues de rojo se tiñe la cura a mi único miedo. \\
Finalmente persigo el deseo de empezar de nuevo. \\
Que no empañe lo que artifició mis tardes, el recuerdo de vivir mucho por morir viejos.\\
Andube en los parques con lo más bonito que tengo, 
un jardín de ajos que no le hacía falta nada. \\


%------------------------------------------------
}
\end{verse}

%------------------------------------------------

\poemauthorright{FUNDACIÓN DE FUENTE LIBRE WAAJACU} % Right-aligned author

%----------------------------------------------------------------------------------------


\end{document}
